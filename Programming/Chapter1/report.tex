\documentclass[a4paper]{article}
\usepackage[affil-it]{authblk}
\usepackage[backend=bibtex,style=numeric]{biblatex}
\usepackage{graphicx} % Required for inserting images
\usepackage{ctex}
\usepackage{epstopdf}
\usepackage{amsfonts,amssymb}
\usepackage{tikz}
\usetikzlibrary{chains}
\usepackage{listings}
\usepackage{xcolor}
\usepackage{float}
\usepackage{hyperref}

\usepackage{amsmath}
%\numberwithin{equation}{section}
%\numberwithin{figure}{section}
\usepackage{chngcntr}
\counterwithout{equation}{section}
\counterwithout{figure}{section}

\usepackage{geometry}
\geometry{margin=1.5cm, vmargin={0pt,1cm}}
\setlength{\topmargin}{-1cm}
\setlength{\paperheight}{29.7cm}
\setlength{\textheight}{25.3cm}


\begin{document}
% =================================================
\title{NA programming homework \#1}

\author{陈澎 3220103443
  \thanks{Electronic address: \texttt{cpzju@zju.edu.cn}}}
\affil{信息与计算科学, 信计2201, Zhejiang University }


\date{\today}

\maketitle

% ============================================
\section*{Problem A}
Function.hpp designs the class "Function" and implements derivative function by using central difference. it also implements the class "polynomial".

EquationSolver.hpp designs an abstract base class "EquationSolver" and its derived class "Bisection\_Method", "Newton\_Method", and "Secant\_Method".

\section*{Problem B}
Compile and run "B.cpp". \par
Bisection method:\\
Solving $x^{-1} - \tan x$ on $[0, \pi/2]$, 
a root is: 0.860334. \\
Solving $x^{-1} - 2^x$ on $[0, 1]$, 
a root is: 0.641186. \\
Solving $2^{-x} - e^x + 2\cos x - 6$ on $[1, 3]$, 
a root is: 1.82938. \\
Solving $(x^3+4x^2+3x+5)/(2x^3-9x^2+18-2)$ on $[0, 4]$, 
a root is: 0.117877. \\

\section*{Problem C}
Compile and run 'C.cpp'. \par
Newton's method:\\
Solving $x = \tan x$ near 4.5 and 7.7,
a root near 4.5 is: 4.49341 and a root near 7.7 is: 7.72525.

\section*{Problem D}
Compile and run 'D.cpp'. \par
Secant method: \\
Solving $\sin(x/2)-1$\\
With initial values 0 and $\pi/2$, the root is: 3.14153. \\
With initial values $2\pi/3$ and $3\pi/4$, the root is: 3.14152.\\
Solving $e^x-\tan(x)$\\
With initial values 1 and 1.4, the root is: 1.30633.\\
With initial values 1.2 and 1.6, the root is: 1.30633.\\
Solving $x^3-12x^2+3x+1$\\
With initial values 0 and -0.5, the root is: -0.188685.\\
With initial values 0.2 and -0.7, the root is: -0.188685. \par
The reason why it gets different result in solving $\sin(x/2)-1$ may be that
the secant method requires two initial estimates. If these values are too far from the actual root or too close together (which
 can lead to numerical instability), it may converge slowly or even diverge.

\section*{Problem E}
Compile and run 'E.cpp'. \par
Solved by Bisection Method, with initial section $[0.1,0.2]$, the depth is: 0.17ft.\par
Solved by Newton Method, with initial value 0.2, the depth is: 0.17ft.\par
Solved by Secant Method, with initial values 0.1 and 0.15 the depth is: 0.17ft.\par


\section*{Problem F}
Compile and run 'F.cpp'. \par
(a) Solved by Newton method,
with initial value $\pi/6$, the root is: 32.9722°.\par
(b) Solved by Newton method,
with initial value 33° but D = 30, the root is: 33.1689°.\par
(c) Solved by secant method.\\
With initial value 33° and 90°, The root is: 33.1689°.\\
With initial value 33° and 150°, The root is: 168.5°.\\
With initial value 33° and 145°, The root is: 146.831°.\par

The reason why it gets different result may be that
the secant method requires two initial estimates. If a function has multiple roots, depending on your starting points, 
the method may converge to different roots. If these initial values are too far from the actual root or too close together (which can lead 
to numerical instability), it may converge slowly or even diverge.
If these values are too far from the actual root or too close together (which
 can lead to numerical instability), it may converge slowly or even diverge. 
 The secant method may converge to a root very slowly or not at all if the secant lines become nearly horizontal (i.e., the function's derivative is very small). This can happen especially near inflection points or local minima/maxima.
\end{document}