\documentclass[a4paper]{article}
\usepackage[affil-it]{authblk}
\usepackage{graphicx} % Required for inserting images
\usepackage{ctex}
\usepackage{epstopdf}
\usepackage{amsmath}
\usepackage{amsfonts,amssymb}
\usepackage{tikz}
\usetikzlibrary{chains}
\usepackage{listings}
\usepackage{xcolor}
\usepackage{float}
\usepackage{hyperref}
\usepackage{geometry}
\geometry{margin=1.5cm, vmargin={0pt,1cm}}
\setlength{\topmargin}{-1cm}
\setlength{\paperheight}{29.7cm}
\setlength{\textheight}{25.3cm}


\begin{document}
% =================================================
\title{NA theoretical homework \#2}

\author{陈澎 Chen Peng 3220103443
  \thanks{Email: \texttt{cpzju@zju.edu.cn}}}
\affil{Xinji 2201, Zhejiang University }


\date{\today}

\maketitle

% ============================================
\section  *{I.}
\subsection*{I-a.}
We have
$$
\begin{aligned}
    p_1(f; x) &= f(1) + (f(2) - f(1)) \frac{x - 1}{2 - 1}   \\
    &= 1 + (\frac{1}{2} - 1) \frac{x - 1}{2 - 1}    \\
    &= -\frac{1}{2}x + \frac{3}{2}, \ x\in(1,2)
\end{aligned}
$$
and $f''(\xi(x)) = \dfrac{2}{(\xi(x))^3}$. After substituting, we have
$$
\begin{aligned}
    \frac{1}{x} - (-\frac{1}{2}x + \frac{3}{2}) &=\frac{1}{\xi(x)^3} (x - 1) (x - 2),    \\
\end{aligned}
$$
which induces that $\xi(x)=(2x)^{\frac{1}{3}},\  x\in(1,2).$

\subsection*{I-b.}
According to the result in question I-a, we extend the domain of $\xi(x)$ from $(1,2)$ to $[1,2]$ and have
$$
  \xi(x)=(2x)^{\frac{1}{3}},\  x\in[1,2].
$$ 
Since $\xi(x)$ is increasing in $[1, 2]$, it is clear that $$\max_{x \in [1, 2]}\xi(x)=\xi(2)=2^\frac{2}{3}$$ and $$\min_{x \in [1, 2]}\xi(x)=\xi(1)=2^\frac{1}{3}.$$

$f''(\xi(x)) = \frac{2}{(\xi(x))^3} = \frac{1}{x}$ is decreasing in $[1, 2]$, so

$$
\max_{x \in [1, 2]} f''(\xi(x)) = f''(\xi(1)) = 1.
$$


\section*{II.}
Set
$$
L_{i} = \left[\prod_{\substack{ 0 \leq j \leq n \\ j \neq i}} \frac{x-x_j}{x_i-x_j} \right]^2, \forall i = 0, 1, \dots, n.
$$

It is clear that $\forall i, j = 0, 1,\dots, n,\ i \not = j,\ L_{i}(x_i) = 1,\ L_{i}(x_j) = 0,$ and $L_{i} \in \mathbb{P}^{+}_{2n}$.

So 
$$
p(x) = \sum_{i=0}^{n} f(x_i) L_{i}(x)
$$ 
is the polynomial we want.

\section*{III.}

\subsection*{III-a.}
Prove by induction.

For $n = 0$, we have
$$
  f[t] = e^t,
$$
which satisfies the formula since $\frac{(e - 1)^0}{0!} e^t = e^t$.

Assume the formula holds for $n = k$, i.e.,
$$
  f[t, t + 1, \ldots, t + k] = \frac{(e - 1)^k}{k!} e^t.
$$

We need to show it holds for $n = k + 1$.

By the definition of divided differences, we have
$$
  f[t, t + 1, \ldots, t + k, t + k + 1] = \frac{f[t + 1, t + 2, \ldots, t + k + 1] - f[t, t + 1, \ldots, t + k]}{(t + k + 1) - t}.
$$

Using the inductive hypothesis, we get
$$
  f[t + 1, t + 2, \ldots, t + k + 1] = \frac{(e - 1)^k}{k!} e^{t + 1},
$$
and
$$
  f[t, t + 1, \ldots, t + k] = \frac{(e - 1)^k}{k!} e^t.
$$

After substituting, we have
$$\begin{aligned}
  f[t, t + 1, \ldots, t + k, t + k + 1] &= \dfrac{\dfrac{(e - 1)^k}{k!} e^{t + 1} - \dfrac{(e - 1)^k}{k!} e^t}{k + 1}\\
  &= \frac{(e - 1)^{k + 1}}{(k + 1)!} e^t.
\end{aligned}
$$

Thus, the formula holds for $n = k + 1$. By induction, the formula is true for all $n \in \mathbb{N}$.

\subsection*{III-b.}
From Corollary 2.22, we have
$$
\exists \xi\in(0,n)\ \text{s.t.}\  f[0, 1, \ldots, n] = \frac{1}{n!} f^{(n)}(\xi).
$$

Because $f^{(n)}(x) = e^x$, we have
$$
  f[0, 1, \ldots, n] = \frac{1}{n!} e^\xi.
$$

From III-a, we have:
$$
  f[0, 1, \ldots, n] = \frac{(e - 1)^n}{n!}.
$$

After substituting, we have
$$
  \frac{(e - 1)^n}{n!} = \frac{1}{n!} e^\xi,
$$
which induces that
$$
 \xi = n \ln(e - 1).
$$

Since 
$$
\begin{aligned}
  \dfrac{\xi}{\frac{n}{2}} &= 2 \ln(e-1) \\
  &= \ln(e - 1)^2 \\
  &> \ln 1.7^2 \\
  &= \ln 2.89 \\
  &> 1,
\end{aligned}
$$
it induces that $\xi > \frac{n}{2}$.
   
Therefore, $\xi$ is located to the right of the midpoint $n/2$.


\section*{IV.}

\subsection*{IV-a.}
Calculate the table of divided differences:

\begin{tabular}{c|cccc}
  $x_0 = 0$ & 5 &   &  &   \\
  $x_1 = 1$ & 3 & -2 &  &   \\
  $x_2 = 3$ & 5 & 1 & 1 &   \\
  $x_3 = 4$ & 12 & 7 & 2 & 0.25 \\
\end{tabular}

Therefore, the Newton interpolation polynomial is:
$$
\begin{aligned}
  p_3(f;x) &= 5 - 2(x - 0) + 1(x - 0)(x - 1) + 0.25(x - 0)(x - 1)(x - 3)\\
  &= 5 - 2x + x(x - 1) + 0.25x(x - 1)(x - 3).
\end{aligned}
$$
  

\subsection*{IV-b.}
$$
\begin{aligned}
  p_3'(f;x) &= -2 + x - 1 + x + 0.25 \left[ (x - 1)(x - 3) + x(x - 3) + x(x - 1) \right]   \\
  &= 0.75x^2 - 2.25.
\end{aligned}
$$

Let $p_3'(f;x) = 0$, we have
$$
  x = \sqrt{3} \in (1, 3).
$$

Thus $p_3'(f;x)$ is decreasing in $(1,\sqrt{3})$ and increasing in $(\sqrt{3},3)$

Therefore, the approximate location of the minimum $x_{\text{min}}$ is $\sqrt{3}$.

\section*{V.}
\subsection*{V-a.}
Calculate the table of divided differences:
\begin{tabular}{c|cccccc}
  $0$ & 0   &     &    &    &   &   \\
  $1$ & 1   & 1   &    &    &   &   \\
  $1$ & 1   & 7   & 6  &    &   &   \\
  $1$ & 1   & 7   & 21 & 15 &   &   \\
  $2$ & 128 & 127 & 120& 99 & 42&   \\
  $2$ & 128 & 448 & 321& 201&102& 30\\
\end{tabular}

Therefore
$$
f[0,1,1,1,2,2] = 30.
$$

\subsection*{V-b.}

The 5th derivative of f is $f^{(5)}(x) = 2520x^2$.

Let
$$
\frac{1}{5!}f^{(5)}(\xi) = f[0, 1, 1, 1, 2, 2] = 30, \ \xi\in(0,2).
$$

We have  $\xi =\frac{\sqrt{70}}{7}\approx 1.195$.

\section*{VI.}
\subsection*{VI-a.}
Calculate the table of divided differences:
$$
\begin{tabular}{c|ccccc}
    $x_0 = 0$ & 1 &   &   &       & \\
    $x_1 = 1$ & 2 & 1 &   &       & \\
    $x_2 = 1$ & 2 & -1& -2&       & \\
    $x_3 = 3$ & 0 & -1& 0 & $\frac{2}{3}$& \\
    $x_4 = 3$ & 0 & 0 &0.5& 0.25 & $-\frac{5}{36}$ \\
\end{tabular}
$$

Therefore, the Hermite interpolation polynomial is

$$
\begin{aligned}
    p(x) &= 1 + x -2x(x - 1) + \frac{2}{3}x(x-1)^2 -\frac{5}{36}x(x-1)^2(x-3).    \\
\end{aligned}
$$

So
$$
f(2) \approx p(2) \approx 0.6111.
$$

\subsection*{VI-b.}

By Theorem 2.37, we have 
$$
\exists \xi_{x} \in (0,3)\ \text{s.t.}\ R_{4}(f;x)=f(x) - p(f; x) = \frac{f^{(5)}(\xi_{x})}{5!} x(x-1)^{2}(x-3)^2.
$$

Because $f^{(5)}(x)\leq M$ on $[0,3]$, we have
$$
\begin{aligned}
  \lvert R_{4}(f;2)\rvert &= \lvert \frac{f^{(5)}(\xi_{2})}{5!} (2 - 0)(2 - 1)^2(2-3)^2 \rvert \\
  &\leq \frac{M}{60}.
\end{aligned}
$$

Therefore the maximum possible error of the above answer is $\frac{M}{60}$.

\section*{VII.}
\subsection*{VII-a.}
Prove by induction.

For $n = 1$, we have
$$
  \Delta f(x)=f(x+h)-f(x)
$$
which satisfies the formula since 
$$1!h^{1}f[x_0,x_1]=h\frac{f(x_1)-f(x_0)}{x_1-x_0}=f(x+h)-f(x).$$

Assume the formula holds for $n = k$, i.e.,
$$
  \Delta^k f(x)=k!h^{k} f[x_0,x_1,\ldots,x_k].
$$

We need to show it holds for $n = k + 1$.

By the definition, we have
$$
\begin{aligned}
  \Delta^{k+1} f(x)&=\Delta^k f(x+h)-\Delta^k f(x) \\
  &=k!h^{k} f[x_1,x_2,\ldots,x_{k+1}]-k!h^{k} f[x_0,x_1,\ldots,x_k]\\
  &=k!h^{k} (x_{k+1}-x_0)\frac{f[x_1,x_2,\ldots,x_{k+1}]-f[x_0,x_1,\ldots,x_k]}{x_{k+1}-x_0}\\
  &=(k+1)!h^{k+1}f[x_0,x_1,\ldots,x_k,x_{k+1}].
\end{aligned}
$$

Thus, the formula holds for $n = k + 1$. By induction, the formula is true for all $n \in \mathbb{N}^*$.

\subsection*{VII-b.}
Prove by induction.

For $n = 1$, we have
$$
  \nabla f(x)=f(x)-f(x-h),
$$
which satisfies the formula since
$$1!h^{1}f[x_0,x_{-1}]=h\frac{f(x_0)-f(x_{-1})}{x_0-x_{-1}}=f(x)-f(x-h).$$

Assume the formula holds for $n = k$, i.e.,
$$
\nabla^{k} f(x)=k!h^{k} f[x_0,x_{-1},\ldots,x_{-k}].
$$

We need to show it holds for $n = k + 1$.

By the definition of divided differences, we have
$$
\begin{aligned}
  \nabla^{k+1} f(x)&=\nabla^{k} f(x)-\nabla^{k} f(x-h)\\
  &=k!h^{k} f[x_0,x_{-1},\ldots,x_{-k}] - k!h^{k} f[x_{-1},x_{-2}\ldots,x_{-k-1}]\\
  &=k!h^{k} (x_0-x_{-k-1})\frac{f[x_0,x_{-1},\ldots,x_{-k}] - f[x_{-1},x_{-2}\ldots,x_{-k-1}]}{x_0-x_{-k-1}}\\
  &=(k+1)!h^{k+1}f[x_0,x_{-1},\ldots,x_{-k},x_{-k-1}]
\end{aligned}
$$

Thus, the formula holds for $n = k + 1$. By induction, the formula is true for all $n \in \mathbb{N}^*$.

\section*{VIII.}
\subsection*{VIII-a.}
Prove by induction.

For $n = 0$, we have
$$
\frac{\partial}{\partial x_0} f[x_0] = f'(x_0) = f[x_0, x_0]
$$
which satisfies the formula.

Assume the formula holds for $n = k$, i.e.,
$$
\frac{\partial}{\partial x_0} f[x_0, x_1, \dots, x_k] = f[x_0, x_0, x_1, \dots, x_k].
$$

We need to show it holds for $n = k + 1$.

By the definition, we have
$$ 
\begin{aligned}
  &\frac{\partial}{\partial x_0} f[x_0, x_1, \dots, x_{k+1}] \\
  =& \frac{\partial}{\partial x_0} \frac{f[x_1, x_2, \dots, x_{k+1}] - f[x_0, x_1, \dots, x_{k}]}{x_{k+1} - x_0}\\
  =& \frac{f[x_1, x_2, \dots, x_{k+1}] - f[x_0, x_1, \dots, x_{k}]}{(x_{k+1}-x_0)^2} - \frac{1}{x_{k+1}-x_0} \frac{\partial}{\partial x_0} f[x_0, x_1, \dots, x_{k}].\\
\end{aligned}
$$

Using the inductive hypothesis, we have
$$
\frac{\partial}{\partial x_0} f[x_0, x_1, \dots, x_k] = f[x_0, x_0, x_1, \dots, x_k].
$$

Therefore,
$$
\begin{aligned}
  &\frac{\partial}{\partial x_0} f[x_0, x_1, \dots, x_{k+1}]\\
  =& \frac{1}{x_{k+1}-x_0} \frac{f[x_1, x_2, \dots, x_{k+1}] - f[x_0, x_1, \dots, x_{k}]}{x_{k+1}-x_0} - \frac{1}{x_{k+1}-x_0} f[x_0, x_0, x_1, \dots, x_k]\\
  =& \frac{1}{x_{k+1}-x_0} (f[x_0, x_1, \dots, x_{k+1}] - f[x_0, x_0, x_1, \dots, x_k])\\
  =& f[x_0, x_0, x_1, \dots, x_{k+1}].
\end{aligned}
$$

Thus, the formula holds for $n = k + 1$. By induction, the formula is true for all $n \in \mathbb{N}$.


\subsection*{VIII-b.}
For the partial derivative with respect to one of other variables, for example, $x_i$, we have
$$
  \frac{\partial}{\partial x_i} f[x_0, \dots, x_n] = \frac{\partial}{\partial x_i} f[x_i, x_0, x_1, \dots, x_{i-1}, x_{i+1}, \dots, x_n]
$$

After simple substitution of subscripts, i.e.
$$
  x_{k}=\left\{\begin{aligned}
  &y_{k+1},\ k=0,1,\ldots,i-1,\\
  &y_0,\ k=i,\\
  &y_{k},\ k=i+1,i+2,\ldots,n,
\end{aligned}\right.
$$
and using conclusion from VIII-a
$$
  \frac{\partial}{\partial y_0} f[y_0, y_1, y_2, \dots, y_n]=f[y_0, y_0, y_1, y_2, \dots, y_n],
$$
we have
$$
\begin{aligned}
  \frac{\partial}{\partial x_i} f[x_0, \dots, x_n] &= \frac{\partial}{\partial y_0} f[y_0, y_1, y_2, \dots, y_n]  \\
  &=f[y_0, y_0, y_1, y_2, \dots, y_n]\\
  &=f[x_0, \dots, x_{i-1}, x_i, x_i, x_{i+1}, \dots, x_n].
\end{aligned}
$$

Therefore, the conclusion is similar for the partial derivative with respect to $x_i$, i.e.
$$
\frac{\partial}{\partial x_i} f[x_0, \dots, x_n] = f[x_0, \dots, x_{i-1}, x_i, x_i, x_{i+1}, \dots, x_n].
$$

\section*{IX.}

Set
$$p(x) = a_{0}x^n + a_{1} x^{n-1} + \cdots + a_n$$ 
and 
$$\xi = -1 + 2\frac{x - a}{b - a},$$ 
i.e.
$$x = a+\frac{b-a}{2}(\xi+1)$$
where $x\in [a, b]$.

So
$$
  \max_{x \in [a,b]} \lvert p(\xi)\rvert = \max_{\xi\in[-1,1]} \left|p\left(a+\frac{b-a}{2}(\xi+1)\right)\right|.
$$

Set
$$
P(\xi) = \frac{2^n}{a_0 ({b-a})^n} p\left(a+\frac{b-a}{2}(\xi+1)\right).
$$
It is a polynomial of $\xi$ of degree $n$ with leading coefficient 1.

By Theorem 2.47, we have
$$
\min \max_{\xi\in[-1,1]} \lvert P(\xi)\rvert = \max_{\xi\in[-1,1]} \left|\frac{T_n(\xi)}{2^{n-1}}\right| = \frac{1}{2^{n-1}}.
$$

So
$$
\begin{aligned}
   \min \max_{x \in [a,b]} |p(x)| &=  \frac{\lvert a_0\rvert ({b-a})^n}{2^n} \min\max_{\xi\in[-1,1]} \lvert P(\xi)\rvert\\
   &= \frac{\lvert a_0\rvert ({b-a})^n}{2^n} \frac{1}{2^{n-1}}  \\
   &= \frac{\lvert a_0\rvert ({b-a})^n}{2^{2n-1}},
\end{aligned}
$$
i.e.

$$
\min\max_{x\in[a,b]}\lvert a_0x^n + a_1 x^{n-1} + \cdots + a_n\rvert = \frac{\lvert a_0\rvert ({b-a})^n}{2^{2n-1}}.
$$

\section*{X.}
Assume the conclusion doesn't hold, i.e.

$$
\exists p \in \mathbb{P}^a_{n},\ \mathrm{s.t.} \ \lVert \hat{p}_{n} \rVert_{\infty} > \lVert p \rVert_{\infty}.
$$

Set $P(x) = \hat{p}_n(x) - p(x)$ and $x_k = \cos \frac{k}{n}\pi\in[-1,1],\ k= 0, 1, \dots, n$, then
$$
P(x_k) = (-1)^{k}\lVert\hat{p}_n \rVert_{\infty} - p(x_k),\ k= 0, 1, \dots, n.
$$

By the above assumption, $P(x)$ has alternating signs at these $n+1$ points. Hence $P(x)$ must have n zeros in $[-1, 1]$.
Besides, $P(a) = \hat{p}(a) - p(a) = 1 - 1 = 0, (a > 1)$. So $P(x)$ has at least $n+1$ zeros. 

However by the construction of $P(x)$, the degree of $P(x)$ is at most $n$. Therefore, $P(x)\equiv0$ and $\hat{p}(a) = p(a)$.
This is a contradiction to the assumption.

So, the original statement must be true, i.e.
$$\forall p\in\mathbb{P}_n^a,\quad\|\hat{p}_n\|_\infty\leq\|p\|_\infty.$$

\section*{XI.}
By Definition 2.50. about the Bernstein base polynomials, 
$$
\forall k = 0, 1, \dots n,\ b_{n, k}(t):=\binom{n}{k}t^k(1-t)^{n-k}.
$$

The right hand side of the formula can be simplied as 
$$
\begin{aligned}
  &\frac{n-k}{n}b_{n,k}(t)+\frac{k+1}{n}b_{n,k+1}(t) \\
  =&\frac{n-k}{n} \frac{n!}{k!(n-k)!} t^k(1-t)^{n-k} + \frac{k+1}{n} \frac{n!}{(k+1)!(n-k-1)!} t^{k+1}(1-t)^{n-k-1} \\
  =&\frac{(n-1)!}{k!(n-k-1)!} t^k(1-t)^{n-k} +\frac{(n-1)!}{k!(n-k-1)!} t^{k+1}(1-t)^{n-k-1} \\
  =&\binom{n-1}{k} t^k(1-t)^{n-k-1}[(1 - t) + t]   \\
  =&\binom{n-1}{k} t^k(1-t)^{n-k-1}  \\
  =&b_{n-1, k}(t).
\end{aligned}
$$

So the formula holds true.

\section*{XII.}
By Definition 2.50. about the Bernstein base polynomials, we have
$$
\forall k = 0, 1, \dots n,\ b_{n, k}(t):=\binom{n}{k}t^k(1-t)^{n-k}.
$$

The left hand side of the formula can be simplied as 
$$
\begin{aligned}
  \int_{0}^{1} b_{n,k}(t) \mathrm{d}t =& \int_{0}^{1} \binom{n}{k}t^k(1-t)^{n-k} \mathrm{d}t \\
  =& \binom{n}{k} \int_{0}^{1} (1-t)^{n-k} \mathrm{d}\frac{t^{k+1}}{k+1} \\
  =& \frac{n!}{(k+1)!(n-k)!} \int_{0}^{1} (1-t)^{n-k} \mathrm{d}t^{k+1} \\
  =& \frac{n!}{(k+1)!(n-k)!} \left[ \left.(1-t)^{n-k}t^{k+1} \right|_{0}^{1} + (n-k)\int_{0}^{1} (1-t)^{n-k-1}t^{k+1} \mathrm{d}t  \right] \\
  =& \frac{n!}{(k+1)!(n-k-1)!} \int_{0}^{1}(1-t)^{n-k-1}t^{k+1}\mathrm{d}t  \\
  =& \int_{0}^{1} \binom{n}{k+1}t^{k+1}(1-t)^{n-k-1}\mathrm{d}t. \\
\end{aligned}
$$

Similarly, we have
$$
  \begin{aligned}
    \int_{0}^{1} b_{n,k}(t) \mathrm{d}t
  =&\int_{0}^{1} \binom{n}{k}t^k(1-t)^{n-k} \mathrm{d}t \\
  =&\int_{0}^{1} \binom{n}{k+1}t^{k+1}(1-t)^{n-k-1}\mathrm{d}t \\
  =&\int_{0}^{1} \binom{n}{k+2}t^{k+2}(1-t)^{n-k-2}\mathrm{d}t \\
  =&\dots \\
  =&\int_{0}^{1} \binom{n}{k+n-k}t^{k+n-k}(1-t)^{n-n+k}\mathrm{d}t \\
  =&\int_0^1t^n\mathrm{d}t \\
  =&\frac{1}{n+1}.
  \end{aligned}
$$

So the formula holds true.

\end{document}